\hypertarget{index_Introduction}{}\section{Introduction}\label{index_Introduction}
Radiative process is one of the most complex and computational intensive part of all model physics. As an essential part of model physics, it directly and indirectly connects all physics processes with model dynamics, and regulates the overall earth-\/atmosphere energy exchanges and transformations. The radiation package in N\+E\+MS physics has standardized component modules. The schematic radiation module structure is shown in table 1.



Radiation parameterizations are intended to provide a fast and accurate method of determined the total radiative flux at any given location. These calculations provide both the total radiative flux at the ground surface, which is needed for the surface energy budget, and the vertical radiative flux divergence, which is used to calculate the radiative heating and cooling rates of a given atmospheric volume. The magnitude of the terms in the surface energy budget can set the stage for moist deep convection and are crucial to the formation of low-\/level clouds. In addition, the vertical radiative flux divergence can produce substantial cooling, particularly at the tops of clouds, which can have strong dynamic effect on cloud evolution.\hypertarget{index_mainpage-components}{}\section{Radiation Scheme Modules}\label{index_mainpage-components}
The following links take you to more information about each module.
\begin{DoxyItemize}
\item Driver Module\+: module\+\_\+radiation\+\_\+driver
\item Shortwave(\+S\+W) Module\+: module\+\_\+radsw\+\_\+main
\item Longwave(\+L\+W) Module\+: module\+\_\+radlw\+\_\+main
\item Astronomy Module\+: module\+\_\+radiation\+\_\+astronomy
\item Aerosol Module\+: module\+\_\+radiation\+\_\+aerosols
\item Cloud Module\+: module\+\_\+radiation\+\_\+clouds
\item Surface Module\+: module\+\_\+radiation\+\_\+surface
\item Gases Module\+: module\+\_\+radiation\+\_\+gases
\end{DoxyItemize}\hypertarget{index_References}{}\section{References}\label{index_References}
Barker, H. W., et al., 2003\+: Assessing 1D atmospheric solar radiative transfer models\+: interpretation and handling of unresolved clouds. J. Clim., 16, 2676-\/2699.

Briegleb, B. P., 1992\+: Delta-\/\+Eddington approximation for solar radiation in the N\+C\+AR community climate model. J. Geophys. Res., 97, 7603-\/7612.

Briegleb, B. P., P. Minnus, V. Ramanathan, and E. Harrison, 1986\+: Comparison of regional clear-\/sky albedo inferred from satellite observations and model computations. J. Clim. and Appl. Meteo., 25, 214-\/226.

Chin, M., R. B. Rood, S-\/J. Lin, J-\/F. Mller, and A. M. Thompson, 2000\+: Atmospheric sulfur cycle simulated in the global model G\+O\+C\+A\+RT\+: Model description and global properties. J. Geophys. Res., 105, 24671-\/24687.

Chou, M. D., M. J. Suarez, C. H. Ho, M. M. H. Yan, and K. T. Lee, 1998\+: Parameterizations for cloud overlapping and shortwave single scattering properties for use in general circulation and cloud ensemble models. J. Clim., 11, 202-\/214.

Clough, S. A., and M. J. Iacono, 1995\+: Line-\/by-\/line calculation of atmospheric fluxes and cooling rates\+: 2. Application to carbon dioxide, ozone, methane, nitrous oxide and the halocarbons. J. Geophys. Res.\+100, 16519-\/16535.

Clough, S. A., M. W. Shephard, E. J. Mlawer, J. S. delamere, M. J. Iacono, K. Cady-\/\+Pereira, S. Boukabara, and P. D. Brown, 2005\+: Atmospheric radiative transfer modeling\+: a summary of the A\+ER codes, J. Quant. Spectrosc. Radiat. Transfer, 91, 233-\/244.

Coakley, J. A., R. D. Cess, and F. B. Yurevich, 1983\+: The effect of tropospheric aerosols on the earth\textquotesingle{}s radiation budget\+: a parameterization for climate models. J. Atmos. Sci., 42, 1408-\/1429.

Fels, S. B., and M.\+D. Schwarzkopf, 1975\+: The simplified exchange approximation\+: A new method for radiative transfer calculations. J. Atmos. Sci., 337, 2265-\/2297.

Frohlich, C. and G. E. Shaw, 1980\+: New determination of Rayleigh scattering in the terrestrial atmosphere. Appl. Opt., 14, 1773-\/1775.

Fu, Q., 1996\+: An accurate parameterization of the solar radiative properties of cirrus clouds for climate models. J. Clim., 9, 2058-\/2082.

Fu, Q., P. Yang, and W.\+B. Sun, 1998\+: An accurate parameterization of the infrared radiative properties of cirrus clouds for climate models. J. Clim., 11, 2223-\/2237.

Hess, M., P. Koepke, and I. Schult, 1998\+: Optical properties of aerosols and clouds\+: The software package O\+P\+AC. Bull. Am. Meteor. Soc., 79, 831-\/844.

Heymsfield, A. J., and G. M. Mc\+Farquhar, 1996\+: High albedos of cirrus in the tropical Pacific warm pool. J. Atmos. Sci., 53, 2424-\/2451.

Hou, Y-\/T., S. Moorthi, K. Campana, 2002\+: Parameterization of solar radiation transfer in the N\+C\+EP Models. N\+C\+EP Office Note 441, 46pp.

Hu, Y. X., and K. Stamnes. 1993\+: An accurate parameterization of the radiative properties of water clouds suitable for use in climate models. J. Clim., 6, 728-\/742.

Kiehl, J. T., J. J. Hack, G. B. Bonan, B. A. Boville, D. L. Williamson, and P. J. Rasch, 1998\+: The national center for atmospheric research community climate model C\+C\+M3. J. Clim., 11, 1131-\/1149.

Matthews, E., 1985\+: Atlas of Archived Vegetation, Land Use, and Seasonal Albedo Data Sets., N\+A\+SA Technical Memorandum 86199, Goddard Institute for Space Studies, New York.

Mlawer, E. J., S. J. Taubman, P. D. Brown, M. J. Iacono, and S. A. Clough, 1997\+: Radiative transfer for inhomogenerous atmospheres\+: R\+R\+TM, a validated correlated-\/k model for the longwave. J. Geophys. Res., 102, 16663-\/16682.

Oreopoulos, L. and Barker, H. W., 1999\+: Accounting for subgrid-\/scale cloud variability in a multi-\/layer, 1D solar radiative transfer algorithm. Q. J. R. Meteorol. Soc., 125, 301-\/330.

Pincus, R., Barker, H. W. and Morcrette, J.-\/J., 2003\+: A fast, flexible, approximate technique of computing radiative transfer for inhomogeneous clouds. J. Geophys. Res., 108, 4376, doi\+: 10.\+1029/2002\+J\+D003322.

Roberts, R. E., J. A. Selby, and L. M. Biberman, 1976\+: Infrared continuum absorption by atmospheric water vapor in the 8-\/12 micron window. Appl. Optics., 15, 2085-\/2090.

Rodgers, C.\+D., 1968\+: Some extension and applications of the new random model for molecular band transmission. Quart. J. Roy. Meteor. Soc., 94,99-\/102.

Sato, M., J. E. Hansen, M. P. Mc\+Cormick, and J. B. Pollack, 1993\+: Stratospheric aerosol optical depth, 1850-\/1990. J. Geophys. Res., 98, 22987-\/22994.

Slingo, A., 1989\+: A G\+CM parameterization for the shortwave radiative properties pf water clouds. J. Atmos. Sci., 46, 1419-\/1427.

Schwarzkopf, M.\+D., and S. B. Fels, 1985\+: Improvements to the algorithm for computing C\+O2 transmissivities and cooling rates. J. Geophys. Res., 90, 10541-\/10550.

Schwarzkopf, M.\+D., and S. B. Fels, 1991\+: The simplified exchange method revisited\+: An accurate, rapid method for computation of infrared cooling rates and fluxes. J. Geophys. Res., 96, 9075-\/9096.

Staylor, W. F. and A. C. Wilbur, 1990\+: Global surface albedos estimated from E\+R\+BE data. Preprints of the Seventh Conference on Atmospheric Radiation, San Francisco CA, American Meteorological Society, 231-\/236.

Stephens, G. L., 1984\+: The parameterization of radiation for numerical weather prediction and climate models. Mon. Wea. Rev., 112, 826-\/867.

Zdunkowski, W. G., Welsch, R. M., and Korb, G. J., 1980\+: An investigation of the structure of typical 2-\/stream methods for the calculation of solar fluxes and heating rates in clouds. Contrib. Atms. Phys., 53, 215-\/238. 