The longwave (LW) radiation module, which is based on A\+ER Rapid Radiative Transfer Model (R\+R\+TM, Mlawer et al. 1997), calculates fluxes and heating rates for the longwave spectral regime. It uses a correlated-\/k distribution method and a linear-\/in-\/tau transmittance table look-\/up to achieve high accuracy and efficiency. The algorithm contains 140 unevenly distributed intervals (g-\/point) in 16 broad spectral bands. Modeled molecular absorbers are water vapor, carbon dioxide, ozone, nitrous oxide, methane, oxygen, nitrogen, and the common halocarbons. Longwave radiation follows the simplified exchanged method of Fels and Schwarzkopf (1975) and Schwarzkopf and Fels (1991), with calculation over spectral bands associated with carbon dioxide, water vapor, and ozone. Schwarzkopf and Fels (1985) transmission coefficients for carbon dioxide, a Roberts et al. (1976) water vapor continuum, and the effects of water vapor-\/carbon dioxide overlap and of a Voigt line-\/shape correction are included. The Rodgers (1968) formulation is adopted for ozone absorption. The cloud emissivity is calculated from the predicted cloud condensate following the approach of the N\+C\+AR C\+CM (Kiehl et al. (1998), Stephens (1984)). LW module divides the longwave spectral region into 16 bands chosen for their homogeneity of contributing species and radiative transfer properties. The selection of the spectral regions is facilitated by consideration of the Plates of spectral cooling rate as a function of log pressure provided by Clough and Iacono (1995). A maximum-\/random cloud overlapping is used. Cloud liquid/ice water path and effective radius are used for calculation of cloud-\/radiative properties. 