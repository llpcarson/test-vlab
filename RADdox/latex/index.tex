\hypertarget{index_Introduction}{}\section{Introduction}\label{index_Introduction}
Radiative process is one of the most complex and computational intensive part of all model physics. As an essential part of model physics, it directly and indirectly connects all physics processes with model dynamics, and regulates the overall earth-\/atmosphere energy exchanges and transformations. The radiation package in N\+E\+MS physics has standardized component modules. The schematic radiation module structure is shown in table 1.



Radiation parameterizations are intended to provide a fast and accurate method of determined the total radiative flux at any given location. These calculations provide both the total radiative flux at the ground surface, which is needed for the surface energy budget, and the vertical radiative flux divergence, which is used to calculate the radiative heating and cooling rates of a given atmospheric volume. The magnitude of the terms in the surface energy budget can set the stage for moist deep convection and are crucial to the formation of low-\/level clouds. In addition, the vertical radiative flux divergence can produce substantial cooling, particularly at the tops of clouds, which can have strong dynamic effect on cloud evolution.

The shortwave radiation parameterization is based on Chou and Suarez (1999) and was modified by Hou et al.(2002) for the G\+FS. It contains eight spectral bands in the ultraviolet and visible region and one spectral band in the near-\/infrared region. It includes absorption by ozone, water vapor,carbon dioxide, and oxygen. A random-\/maximum cloud overlapping is assumed for radiative transfer calculations in the operational G\+FS. Cloud optical depth is parameterized as a function of the predicted cloud condensate path and the effective radius of cloud particles ( $r_e$). Cloud particle single-\/scattering albedo and asymmetry factors are functions of $r_e$. For water droplets. $r_e$ is fixed at $10\mu m$ over the ocean, and specified as $r_e=min[max(5-0.25T_c , 5),10]\mu m$ over land, where $T_c$ is temperature in degrees Celsius. For ice particles, $r_e$ is an empirical function of ice water content and temperature that follows Heymsfield and Mc\+Farquhar (1996). The radiative effects of rain and snow are not included in the operational G\+FS, but the direct radiative effect of atmospheric aerosols is included. The surface albedo over land varies with the surface type, solar spectral band, and season, and is further adjusted by a solar zenith-\/angle-\/dependent factor for the direct solar beam. When the ground has snow cover the grid-\/mean surface albedo is first computed separately for snow-\/free and snow-\/covered areas, and then combined using a snow-\/cover fraction that depends on the surface roughness and snow depth. Snow albedo depends on the solar zenith angle (Briegleb 1992).

A major change was made in longwave radiation on 28 August 2003. The Geophysical Fluid Dynamics Laboratory (G\+F\+DL) model (Schwarzkopf and Fels 1991) was replaced by the Rapid Radiative Transfer Model (R\+R\+TM; Mlawer et al. 1997). The R\+R\+TM computes longwave absorption and emission by water vapor,carbon dioxide,ozone,cloud particles, and various trace gases including $N_2O$, $CH_4$, $O_2$,and four types of halocarbons\mbox{[}chlorofluorocarbons(\+C\+F\+Cs)\mbox{]}.Aerosol effects are not included.\+For consistency with the earlier G\+F\+DL module, no trace gases are included in the R\+R\+TM for the G\+FS forecasts. The R\+R\+TM uses a correlated-\/k distribution method and a transmittance lookup table that is linearly scaled by optical depth to achieve high accuracy and efficiency. The algorithm contains 140 unevenly distributed intervals in 16 spectral bands. It employs the Clough-\/\+Kneizys-\/\+Davies (C\+K\+D\+\_\+2.\+4) continuum model (Clough et al. 1992) to compute absorption by water vapor at the continuum band. Longwave cloud radiative properties external to the R\+R\+TM depend on cloud liquid/ice water path and the effective radius of ice particles and water droplets (Hu and Stamnes 1993; Ebert and Curry 1992).\hypertarget{index_mainpage-components}{}\section{Radiation Scheme Modules}\label{index_mainpage-components}
The following links take you to more information about each module.
\begin{DoxyItemize}
\item Driver Module\+: \hyperlink{namespacemodule__radiation__driver}{module\+\_\+radiation\+\_\+driver}
\item Shortwave(\+S\+W) Module\+: \hyperlink{namespacemodule__radsw__main}{module\+\_\+radsw\+\_\+main}
\item Longwave(\+L\+W) Module\+: \hyperlink{namespacemodule__radlw__main}{module\+\_\+radlw\+\_\+main}
\item Astronomy Module\+: \hyperlink{namespacemodule__radiation__astronomy}{module\+\_\+radiation\+\_\+astronomy}
\item Aerosol Module\+: \hyperlink{namespacemodule__radiation__aerosols}{module\+\_\+radiation\+\_\+aerosols}
\item Cloud Module\+: \hyperlink{namespacemodule__radiation__clouds}{module\+\_\+radiation\+\_\+clouds}
\item Surface Module\+: \hyperlink{namespacemodule__radiation__surface}{module\+\_\+radiation\+\_\+surface}
\item Gases Module\+: \hyperlink{namespacemodule__radiation__gases}{module\+\_\+radiation\+\_\+gases}
\end{DoxyItemize}\hypertarget{index_component}{}\section{Cloud Properties in Radiation}\label{index_component}
The cloud cover is calculated based on Xu and Randall (1996). \[ \sigma =RH^{k_{1}}\left[1-exp\left(-\frac{k_{2}q_{l}}{\left[\left(1-RH\right)q_{s}\right]^{k_{3}}}\right)\right] \] Where $RH$ is relative humidity, $q_{l}$ is the cloud condensate, $q_{s}$ is saturation specific humidity, $k_{1}(=0.25)$, $k_{2}(=100)$, $k_{3}(=0.49)$ are the empirical parameters. The cloud condensate is partitioned into cloud water and ice in radiation based on temperature. Cloud drop effective radius ranges 5-\/10 microns over land depending on temperature. Ice crystal radius is function of ice water content (Heymsfield and Mc\+Farquhar (1996)). Maximum-\/randomly cloud overlapping is used in both long-\/wave radiation and short-\/wave radiation. Convective clouds are not considered in radiation.\hypertarget{index_component}{}\section{Cloud Properties in Radiation}\label{index_component}

\begin{DoxyItemize}
\item Updated and optimized R\+R\+T\+MG + Neural Net Emulator option
\item Higher frequency of radiation calls (possibly every time step with NN)
\item Uncorrelated cloud overlap \& inhomogeneous water/ice clouds with rain/snow
\item Updated C\+O2 with vertically varying profile
\item Observed estimate of trace gases -\/ prescribed global mean climatology
\item Mean solar constant of 1361 $W/m^2$ (with 11 year solar cycle -\/ Van Den Dool)
\item G\+O\+C\+A\+RT interactive aerosol model, updated with vertical profile
\item Land albedo using M\+O\+D\+IS retrieval based monthly data
\item Ocean albedo based on salinity, surface wind and Cosz
\item Spectrally varing emissivity
\end{DoxyItemize}\hypertarget{index_References}{}\section{References}\label{index_References}
Barker, H. W., et al., 2003\+: Assessing 1D atmospheric solar radiative transfer models\+: interpretation and handling of unresolved clouds. J. Clim., 16, 2676-\/2699.

Briegleb, B. P., 1992\+: Delta-\/\+Eddington approximation for solar radiation in the N\+C\+AR community climate model. J. Geophys. Res., 97, 7603-\/7612.

Briegleb, B. P., P. Minnus, V. Ramanathan, and E. Harrison, 1986\+: Comparison of regional clear-\/sky albedo inferred from satellite observations and model computations. J. Clim. and Appl. Meteo., 25, 214-\/226.

Chin, M., R. B. Rood, S-\/J. Lin, J-\/F. Mller, and A. M. Thompson, 2000\+: Atmospheric sulfur cycle simulated in the global model G\+O\+C\+A\+RT\+: Model description and global properties. J. Geophys. Res., 105, 24671-\/24687.

Chou, M. D., M. J. Suarez, C. H. Ho, M. M. H. Yan, and K. T. Lee, 1998\+: Parameterizations for cloud overlapping and shortwave single scattering properties for use in general circulation and cloud ensemble models. J. Clim., 11, 202-\/214.

Clough, S. A., and M. J. Iacono, 1995\+: Line-\/by-\/line calculation of atmospheric fluxes and cooling rates\+: 2. Application to carbon dioxide, ozone, methane, nitrous oxide and the halocarbons. J. Geophys. Res.\+100, 16519-\/16535.

Clough, S. A., M. W. Shephard, E. J. Mlawer, J. S. delamere, M. J. Iacono, K. Cady-\/\+Pereira, S. Boukabara, and P. D. Brown, 2005\+: Atmospheric radiative transfer modeling\+: a summary of the A\+ER codes, J. Quant. Spectrosc. Radiat. Transfer, 91, 233-\/244.

Coakley, J. A., R. D. Cess, and F. B. Yurevich, 1983\+: The effect of tropospheric aerosols on the earth\textquotesingle{}s radiation budget\+: a parameterization for climate models. J. Atmos. Sci., 42, 1408-\/1429.

Fels, S. B., and M.\+D. Schwarzkopf, 1975\+: The simplified exchange approximation\+: A new method for radiative transfer calculations. J. Atmos. Sci., 337, 2265-\/2297.

Frohlich, C. and G. E. Shaw, 1980\+: New determination of Rayleigh scattering in the terrestrial atmosphere. Appl. Opt., 14, 1773-\/1775.

Fu, Q., 1996\+: An accurate parameterization of the solar radiative properties of cirrus clouds for climate models. J. Clim., 9, 2058-\/2082.

Fu, Q., P. Yang, and W.\+B. Sun, 1998\+: An accurate parameterization of the infrared radiative properties of cirrus clouds for climate models. J. Clim., 11, 2223-\/2237.

Hess, M., P. Koepke, and I. Schult, 1998\+: Optical properties of aerosols and clouds\+: The software package O\+P\+AC. Bull. Am. Meteor. Soc., 79, 831-\/844.

Heymsfield, A. J., and G. M. Mc\+Farquhar, 1996\+: High albedos of cirrus in the tropical Pacific warm pool. J. Atmos. Sci., 53, 2424-\/2451.

Hou, Y-\/T., S. Moorthi, K. Campana, 2002\+: Parameterization of solar radiation transfer in the N\+C\+EP Models. N\+C\+EP Office Note 441, 46pp.

Hu, Y. X., and K. Stamnes. 1993\+: An accurate parameterization of the radiative properties of water clouds suitable for use in climate models. J. Clim., 6, 728-\/742.

Kiehl, J. T., J. J. Hack, G. B. Bonan, B. A. Boville, D. L. Williamson, and P. J. Rasch, 1998\+: The national center for atmospheric research community climate model C\+C\+M3. J. Clim., 11, 1131-\/1149.

Matthews, E., 1985\+: Atlas of Archived Vegetation, Land Use, and Seasonal Albedo Data Sets., N\+A\+SA Technical Memorandum 86199, Goddard Institute for Space Studies, New York.

Mlawer, E. J., S. J. Taubman, P. D. Brown, M. J. Iacono, and S. A. Clough, 1997\+: Radiative transfer for inhomogenerous atmospheres\+: R\+R\+TM, a validated correlated-\/k model for the longwave. J. Geophys. Res., 102, 16663-\/16682.

Oreopoulos, L. and Barker, H. W., 1999\+: Accounting for subgrid-\/scale cloud variability in a multi-\/layer, 1D solar radiative transfer algorithm. Q. J. R. Meteorol. Soc., 125, 301-\/330.

Pincus, R., Barker, H. W. and Morcrette, J.-\/J., 2003\+: A fast, flexible, approximate technique of computing radiative transfer for inhomogeneous clouds. J. Geophys. Res., 108, 4376, doi\+: 10.\+1029/2002\+J\+D003322.

Roberts, R. E., J. A. Selby, and L. M. Biberman, 1976\+: Infrared continuum absorption by atmospheric water vapor in the 8-\/12 micron window. Appl. Optics., 15, 2085-\/2090.

Rodgers, C.\+D., 1968\+: Some extension and applications of the new random model for molecular band transmission. Quart. J. Roy. Meteor. Soc., 94,99-\/102.

Sato, M., J. E. Hansen, M. P. Mc\+Cormick, and J. B. Pollack, 1993\+: Stratospheric aerosol optical depth, 1850-\/1990. J. Geophys. Res., 98, 22987-\/22994.

Slingo, A., 1989\+: A G\+CM parameterization for the shortwave radiative properties pf water clouds. J. Atmos. Sci., 46, 1419-\/1427.

Schwarzkopf, M.\+D., and S. B. Fels, 1985\+: Improvements to the algorithm for computing C\+O2 transmissivities and cooling rates. J. Geophys. Res., 90, 10541-\/10550.

Schwarzkopf, M.\+D., and S. B. Fels, 1991\+: The simplified exchange method revisited\+: An accurate, rapid method for computation of infrared cooling rates and fluxes. J. Geophys. Res., 96, 9075-\/9096.

Staylor, W. F. and A. C. Wilbur, 1990\+: Global surface albedos estimated from E\+R\+BE data. Preprints of the Seventh Conference on Atmospheric Radiation, San Francisco CA, American Meteorological Society, 231-\/236.

Stephens, G. L., 1984\+: The parameterization of radiation for numerical weather prediction and climate models. Mon. Wea. Rev., 112, 826-\/867.

Xu, K. M., and D. A. Randall, 1996\+: A semiempirical cloudiness parameterization for use in climate models. J. Atmos. Sci., 53, 3084-\/3102. Zdunkowski, W. G., Welsch, R. M., and Korb, G. J., 1980\+: An investigation of the structure of typical 2-\/stream methods for the calculation of solar fluxes and heating rates in clouds. Contrib. Atms. Phys., 53, 215-\/238. 